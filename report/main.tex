\documentclass[conference]{IEEEtran}
\IEEEoverridecommandlockouts

\usepackage[numbers]{natbib}
\usepackage{amsmath,amssymb,amsfonts}
\usepackage{graphicx}
\usepackage{textcomp}
\usepackage{xcolor}
\usepackage{url}
\usepackage{booktabs}
\usepackage{listings}

\lstset{
  basicstyle=\ttfamily\footnotesize,
  breaklines=true,
  frame=single,
  captionpos=b
}

\begin{document}

\title{Intelligent Property Price Prediction System\\(Milestone 1 Report)}

\author{
\IEEEauthorblockN{Neural  Hackers}
\IEEEauthorblockA{Real Estate Analytics Project}
}

\maketitle

\begin{abstract}
This report presents a machine learning system for residential property price prediction implemented in the repository \texttt{lomesh2312/Intelligent-Property-Price-Prediction}. The project trains and compares Linear Regression, Decision Tree Regressor, and Random Forest Regressor models on the Kaggle Housing dataset. The best model is selected using RMSE and deployed through a Streamlit interface for interactive prediction. The final system satisfies Milestone 1 requirements for supervised learning, evaluation, and user-facing prediction.
\end{abstract}

\begin{IEEEkeywords}
Property price prediction, real estate analytics, regression, Streamlit, machine learning.
\end{IEEEkeywords}

\section{Introduction}
Accurate property valuation is essential for buyers, sellers, and investors. Manual estimation is often subjective and may ignore important feature interactions. This project develops a data-driven property valuation pipeline using supervised regression models trained on historical housing data. The system is exposed through a web interface where users enter property features and receive instant predicted prices.

\section{Problem Statement}
Given structured housing attributes (e.g., area, bedrooms, bathrooms, amenities, and furnishing status), predict the market price of a property. The system should select a strong-performing regression model and provide a practical interface for end users.

\section{Dataset and Inputs}
\subsection{Dataset}
The implementation uses the Kaggle Housing Prices dataset (\texttt{Housing.csv}) \cite{kaggle_housing}. The target column is \texttt{price}.

\subsection{Input Features}
The training script uses the following engineered features:
\begin{lstlisting}[language={}]
area, bedrooms, guestroom, bathrooms,
mainroad, prefarea, stories, parking,
basement, airconditioning,
furnishingstatus_semi-furnished,
furnishingstatus_unfurnished
\end{lstlisting}

\subsection{Output}
The model outputs a predicted numeric property price. The UI additionally displays the selected best model name.

\section{Methodology}
\subsection{Preprocessing}
The pipeline in \texttt{train\_model.py} performs:
\begin{itemize}
  \item Binary mapping for yes/no columns: \texttt{mainroad}, \texttt{guestroom}, \texttt{basement}, \texttt{airconditioning}, \texttt{prefarea}.
  \item One-hot encoding of \texttt{furnishingstatus} with drop-first.
\end{itemize}

\subsection{Models}
Three supervised regressors are trained \cite{sklearn,breiman2001}:
\begin{itemize}
  \item Linear Regression
  \item Decision Tree Regressor
  \item Random Forest Regressor (200 trees)
\end{itemize}

\subsection{Training and Selection}
Data is split with 80:20 train-test partition (\texttt{random\_state=42}). Each model is evaluated on test data with RMSE, and the best model is selected by minimum RMSE. The selected model and its name are saved using Joblib:
\begin{lstlisting}[language=Python]
joblib.dump((best_model, best_model_name),
            "model/best_house_price_model.pkl")
\end{lstlisting}

\section{Results}
The repository README reports the following example performance:
\begin{table}[ht]
\centering
\caption{Model Performance (Repository Reported)}
\begin{tabular}{lcc}
\toprule
Model & RMSE & $R^2$ \\
\midrule
Linear Regression & 1,347,778.29 & 0.6406 \\
Decision Tree & 1,754,528.50 & 0.3910 \\
Random Forest & 1,409,390.67 & 0.6070 \\
\bottomrule
\end{tabular}
\label{tab:results}
\end{table}

Based on RMSE, Linear Regression is selected as the final deployed model in this implementation.

\section{System Interface and Deployment}
\subsection{Streamlit Interface}
The application (\texttt{property\_price\_prediction/app.py}) provides an input form for property attributes and validates required user selections before inference. Predicted price and selected model name are displayed after submission.

\subsection{Deployment}
The project is publicly deployed at:
\begin{center}
\textbf{\url{https://intelligent-property-price-prediction.streamlit.app/}}
\end{center}

\section{Repository Information}
Source repository:
\begin{center}
\textbf{\url{https://github.com/lomesh2312/Intelligent-Property-Price-Prediction}}
\end{center}

Main structure:
\begin{lstlisting}[language={}]
Intelligent-Property-Price-Prediction/
|- README.md
|- property_price_prediction/
|  |- app.py
|  |- train_model.py
|  |- requirements.txt
|  |- model/
|     |- best_house_price_model.pkl
\end{lstlisting}

\section{Conclusion}
This project delivers a working Milestone 1 property price prediction system with supervised regression, basic preprocessing, model comparison, persisted artifact, and public Streamlit deployment. It provides a practical baseline for extending into an agentic advisory system in subsequent milestones.

\bibliographystyle{IEEEtranN}
\bibliography{references}

\end{document}
